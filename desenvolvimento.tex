%---------- Quinto Capítulo: Desenvolvimento do Software ----------
\chapter{Desenvolvimento do Software}
\label{chap:desenv}

Utilizando a metodologia e o projeto do sistema apresentados nos capítulos \ref{metod} e \ref{specs}, respectivamente, inicou-se o desenvolvimento efetivo do sistema.
Primeiramente foram configurados o ambiente de desenvolvimento Netbeans juntamente com a ferramenta de gerenciamento de projetos e dependências Maven.
Feito isso, deu-se início a implementação dos módulos \emph{Core}, \emph{Importer}, \emph{Web Service} e Cliente.

%wrappers Neo4j

Nas subseções a seguir serão descritos detalhes a respeito da implementação de cada módulo.

\section{Configuração do ambiente de desenvolvimento}

\section{Core}
O \emph{Core} foi organizado de tal forma que todas as suas entidades com dados originários dos arquivos GTFS possuam suas respectivas \emph{factories}.
Desta forma, toda criação de entidades é realizada através de uma \emph{factory}, centralizando este processo a somente uma classe por entidade.

\subsection{Location}

\subsection{Stop}

\subsection{Route}

\subsection{Trip}

\subsection{StopTime}

\subsection{Connection}

\subsection{DatabaseController}


\section{Importer}

\section{Web Service}

\section{Cliente}

\section{Considerações}