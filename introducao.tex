%---------- Primeiro Capitulo: Introdução ----------
\chapter{Introdução}
A situação do trânsito das médias e grandes cidades brasileiras vem se tornando cada vez mais grave.
Tempo e dinheiro são perdidos devido a imensos congestionamentos e, estes por sua vez, resultam em grandes gastos em combustível, manutenção veicular e rodoviária.

Atualmente existe um projeto de lei tramitando no Congresso Nacional, o PL nº 1.687 \cite{FlexaRibeiro2010}, que consiste principalmente em priorizar o transporte público em detrimento ao particular e também incentivar o transporte não-motorizado em detrimento do motorizado.
Em segundo plano, vincula o planejamento urbano ao sistema de transporte, fazendo com que as cidades cresçam com um sistema de transporte ordenado.

Segundo \cite{IPEA2011}, atualmente a cada 12 reais investidos no transporte particular, somente 1 é investido no transporte público.
Este projeto de lei busca inverter as prioridades.
Para ilustrar a cituação atual do sistema de transporte público brasileiro, abaixo estão relacionadas as proporções de carros e ônibus em algumas cidades brasileiras.

\begin{table}[!htb]
	\centering
	\caption{Relação entre carros e ônibus nas principais cidades brasileiras}
	\label{tab:carro_onibus}
	\begin{tabular}{lcccc}
		\hline
		\textbf{Cidade} & \textbf{\% Carros} & \textbf{\% Ônibus} \\
		\hline
		\textbf{Belo Horizonte} & 77 & 23 \\
		\textbf{Brasília} & 91 & 9 \\
		\textbf{Curitiba} & 79 & 21 \\
		\textbf{Porto Alegre} & 69 & 31 \\
		\textbf{Recife} & 84 & 16 \\
		\textbf{Rio de Janeiro} & 74 & 26 \\
		\textbf{São Paulo} & 88 & 12 \\
		\hline
	\end{tabular}
	\fonte{\cite{IPEA2011}}
\end{table}

Este mesmo estudo mostra que o número de veículos usados para transporte individual no Brasil cresce cada vez mais: o número de usuários de carros e notocicletas cresceu 9\% e 19\% ao ano , respectivamente, enquanto isso, a porcentagem do uso de transporte público foi de 68\% para 51\% do total de viagens motorizadas \cite{IPEA2011}.

Considerando a situação atual do sistema de transporte público nacional, este projeto tem como escopo a construção uma ferramenta que auxilie o usuário a planejar rotas através do sistema de transporte público.
O objetivo de usar um sistema automatizado para o planejamento de rotas é encontrar automaticamente a maneira mais rápida de se alcançar o destino apenas através do sistema de transporte público e percorrendo, se necessário, pequenos trechos à pé.
Os sistemas de transporte público são, em virtualmente todos os casos, muito menos cômodos do que utilizar um automóvel para se locomover em grandes cidades, e os que tem acesso a um automóvel raramente utilizam o transporte público. Entretanto, mesmo para essas pessoas, o transporte público pode oferecer algumas vantagens, como economia significativa em combustível, manutenção de veículo e estacionamento. Ao facilitar o planejamento de rotas, esta opção de transporte pode se tornar muito mais cômoda e confortável, já que essa tarefa pode não ser trivial.

\section{Motivação}

Muitas vezes, os meios de transporte público não são utilizados puramente devido ao desconhecimento das linhas de transporte presentes em um determinado centro urbano por parte dos usuários. 

Na maioria das cidades brasileiras não existe um canal de comunicação entre os usuários do serviço de transporte público e os responsáveis pela disponibilidade do mesmo. Quando há tal comunicação, as informações são distribuídas através de sistemas precários e com pouca confiabilidade, sendo que apenas uma pequena parcela da população toma conhecimento.

Com o advento da web e a popularização dos meios computacionais, vários sistemas de informação estão surgindo com diferentes propósitos, inclusive na área de transporte público. Tais sistemas são extremamente úteis no auxílio ao planejamento de rotas, porém necessitam de informações precisas e confiáveis, o que consiste em atualizações constantes de suas bases de dados, dificultanto portanto sua manutenção.

Nesse contexto, este projeto visa facilitar, do ponto de vista do usuário, o planejamento de trajetos entre localidades situadas nos grandes centros urbanos, com o uso de meios de transporte públicos.

\section{Objetivos}

O presente trabalho tem como objetivos:

\begin{itemize}
	\item Desenvolver uma ferramenta, na forma de serviço web, capaz de auxiliar o planejamento de trajetos urbanos com o uso de transporte público;
	\item Desenvolver uma interface na forma de sítio eletrônico para os usuários do serviço;
	\item Distribuir as tecnologias desenvolvidas sob a forma de software livre, o que viabiliza futuras contribuições ao projeto e integrações a diversos outros sistemas.
\end{itemize}

%apresentar os capitulos do documento