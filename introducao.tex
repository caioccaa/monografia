%---------- Primeiro Capitulo: Introdução ----------
\chapter{Introdução}

Este projeto tem como escopo a construção uma ferramenta que auxilie o usuário a planejar rotas através do sistema de transporte público. O objetivo de usar um sistema automatizado para o planejamento de rotas é encontrar automaticamente a maneira mais rápida de se alcançar o destino apenas através do sistema de transporte público e percorrendo, se necessário, pequenos trechos à pé. Os sistemas de transporte público são, em virtualmente todos os casos, muito menos cômodos do que utilizar um automóvel para se locomover em grandes cidades, e os que tem acesso a um automóvel raramente utilizam o transporte público. Entretanto, mesmo para essas pessoas, o transporte público pode oferecer algumas vantagens, como economia significativa em combustível, manutenção de veículo e estacionamento. Ao facilitar o planejamento de rotas, esta opção de transporte pode se tornar muito mais cômoda e confortável, já que essa tarefa pode não ser trivial.

\section{Motivação}

Muitas vezes, os meios de transporte público não são utilizados puramente devido ao desconhecimento das linhas de transporte presentes em um determinado centro urbano por parte dos usuários. Com este trabalho, espera-se facilitar, do ponto de vista do usuário, o planejamento de trajetos entre localidades situadas nos grandes centros urbanos, com o uso de meios de transporte públicos.

\section{Objetivos}

O presente trabalho tem como objetivos:

\begin{itemize}
	\item Desenvolver uma ferramenta, na forma de serviço web, capaz de auxiliar o planejamento de trajetos urbanos com o uso de transporte público;
	\item Desenvolver uma interface na forma de sítio eletrônico para os usuários do serviço;
	\item Distribuir as tecnologias desenvolvidas sob a forma de software livre, possibilitando que futuramente outras pessoas façam contribuições ao produto desenvolvido;
\end{itemize}

