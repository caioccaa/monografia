\chapter{Guia de Desenvolvimento}\label{ape:guia} % FIXME: titulo podre

Este apêndice tem como propósito descrever os passos que podem ser utilizados para obter os arquivos do projeto, reproduzir o ambiente de desenvolvimento, executar o \emph{software} e contribuir com o desenvolvimento do mesmo.
As instruções aqui apresentadas são referentes a uma estação de trabalho utilizando GNU/Linux.
Apesar disso, estas instruções podem também ser utilizadas no sistema operacional Microsoft Windows com poucas modificações.

\section{Obtendo os \emph{Softwares} Necessários}

Para obter, compilar e executar o projeto é necessário que os seguintes \emph{softwares} estejam instalados na estação de trabalho:

\begin{itemize}
% misteriosamente, se eu não colocasse o ~, o texto ficava perto demais do bullet da lista
	\item ~ git, versão 1.7 ou superior.
	\item ~ Java Development Kit 7 ou superior.
	\item ~ Apache Maven, versão 3.0 ou superior.
\end{itemize}

Estes aplicativos podem ser, em geral, obtidos a partir do gerenciador de pacotes da distribuição GNU/Linux.
No caso do Microsoft Windows, é necessário obter estes arquivos a partir da página de cada um dos projetos.

\section{Obtendo os Arquivos do Projeto}

Para obter os arquivos do projeto, basta executar o comando \texttt{clone} do git no repositório do projeto no GitHub:

\lstset{language=bash}
\begin{lstlisting}
git clone git://github.com/onibuscerto/onibuscerto.git
\end{lstlisting}

Ao executar este comando, será criado um diretório \texttt{onibuscerto} com o conteúdo do projeto, conforme a estrutura descrita no Capítulo \ref{chap:desenv}.

\section{Obtendo as Dependências e Compilando o Projeto}

Existem várias formas de se compilar o projeto.
Uma delas, é abrir o projeto em uma IDE que suporte projetos Maven como, por exemplo, a IDE Netbeans.
A outra, é utilizar a ferramenta para linha de comando do próprio Apache Maven.
Para tanto, basta executar o seguinte comando na raiz do diretório \texttt{onibuscerto}:

\lstset{language=bash}
\begin{lstlisting}
mvn install
\end{lstlisting}

Note que este comando irá fazer \emph{download} de todas as dependências do projeto tais como o banco de dados Neo4j, o \emph{framework} de testes JUnit, entre outros.
Sendo assim, o tempo de execução deste comando pode variar de acordo com a velocidade da conexão à Internet sendo utilizada.

Ao concluir o \emph{download} das dependências, o Maven irá automaticamente iniciar a compilação do projeto, após a qual serão executados os testes e o \emph{software} compilado será instalado no repositório local do Maven.
Os testes podem ser executados novamente durante o desenvolvimento a qualquer momento através do comando \texttt{mvn test}.

\section{Importando a Base de Dados}

Duas bases de dados distintas estão presentes no projeto \texttt{onibuscerto-importer}: a primeira, situada no diretório \texttt{onibuscerto-importer/src/test/resources/sample} e apenas utilizada para a execução dos testes, é a base de dados de exemplo da especificação do formato GTFS, a segunda é a base de dados do Bay Area Rapid Transit (BART), da cidade de São Francisco, California.
A base de dados está disponível para uso em \citeonline{bart}.
Para importar a base de dados do BART no banco de dados do \emph{software}, basta executar o seguinte comando a partir do diretório do \texttt{onibuscerto-importer}:

\lstset{language=bash}
\begin{lstlisting}
mvn exec:java -e -Dexec.mainClass="com.onibuscerto.importer.ImporterMain"
\end{lstlisting}

Ao executar este comando, diversas mensagens de depuração serão impressas à saída padrão.
Os arquivos do banco de dados com a base de dados do BART importada serão criados no diretório \texttt{onibuscerto-importer/target/db}.

Após importar a base de dados, é necessário instalar a mesma no projeto \texttt{onibuscerto-service}.
Para tanto, basta copiar o diretório \texttt{onibuscerto-importer/target/db} para o diretório \texttt{onibuscerto-service/target/db}, com o seguinte comando:

\lstset{language=bash}
\begin{lstlisting}
cp -R onibuscerto-importer/target/db onibuscerto-service/target/db
\end{lstlisting}

Ao fim destes passos, o \emph{software} estará devidamente instalado, sendo possível executá-lo com os passos descritos na seção a seguir.

\section{Executando o \emph{Software}}

Para executar o \emph{web service}, basta executar o seguinte comando no diretório \texttt{onibuscerto-service}:

\lstset{language=bash}
\begin{lstlisting}
mvn jetty:run
\end{lstlisting}

Este comando irá automaticamente iniciar uma instância do servidor de aplicações Jetty e instalar o Servlet do projeto no mesmo.
Por fim, o cliente \emph{web} pode ser acessado a partir da URL \texttt{http://localhost:8080/}.
O endereço do \texttt{RouteServlet} será \texttt{http://localhost:8080/route}, sendo que este poderá ser utilizado de acordo com as instruções apresentadas no Apêndice \ref{ape:exemplodeuso}.
