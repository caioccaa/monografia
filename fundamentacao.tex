%---------- Segundo Capítulo: Fundamentação --------------

\chapter{Fundamentação}

\section{O Problema do Menor Caminho Multi-Modal}

% citar esse artigo: http://www.scialert.net/fulltext/?doi=jas.2009.3804.3812&org=11#290480_ja

O problema do menor caminho multi-modal consiste em, dada uma rede multi-modal e dois nós nessa rede, a origem e o destino, encontrar o caminho entre a origem e o destino que totaliza o menor custo.
Em uma rede de transporte multi-modal, os nós correspondem às paradas do sistema de transporte, ou aos pontos de origem de destino do trajeto. As arestas correspondem aos deslocamentos entre os nós da rede, e tem associado um custo em tempo e um modo. Os modem correspondem aos serviços ou meios de deslocamento na rede, como por exemplo caminhada ou uma determinada linha de ônibus.
Os serviços de transporte operam segundo tabelas de horários, e portanto para resolver o problema também é necessária a informação do horário de partida no nó de origem.
Uma rede multi-modal pode ser vista como a superposição de diversas redes mono-modais, com a adição de arestas entre elas para representar o tempo de espera associado à troca de serviços.

% vou escrever mais do multi-objetivo depois
Esse problema pode ser generalizado para o caso onde o interessa não é apenas minimizar o tempo de chegada ao destino, mas também minimizar outros objetivos como o custo total do transporte e o número de trocas de serviços. Esta generalização é o problema do menor caminho multi-objetivo.

\section{NoSQL}
%TODO: citar wikipedia =p

NoSQL, ou ``not only SQL'' (não apenas SQL), é uma classe que agrega vários \sigla{SGBD}{Sistema Gerenciador de Banco de Dados}s com características que os diferem dos SGBDs relacionais comuns.
Esses sistemas usualmente permitem que dados sejam armazenados de forma menos rígida em comparação a um esquema de tabelas do modelo relacional, e também nem sempre procuram garantir as propriedades ACID (Atomicidade, Consistência, Isolamento e Durabilidade) comuns nas transações de bancos de dados.
O objetivo desses sistemas é proporcionar um melhor desempenho em algumas aplicações específicas, por exemplo em situações onde existe um grande volume de dados ou de requisições. Esse ganho de desempenho decorre da boa escalabilidade horizontal desses sistemas, ou do fato de eles evitarem junções (produtos cartesianos) entre tabelas.

Existem várias categorias de sistemas NoSQL, de acordo com a implementação e o modo como os dados são armazenados, como por exemplo bancos de dados de documentos, bancos de dados orientados a objetos ou bancos de dados de grafos.
Destes, apenas os bancos de dados de grafos faz parte do escopo do trabalho.

\subsection{Bancos de Dados de Grafos}

%TODO: referências(wiki)

Bancos de dados de grafos são bancos de dados que armazenam os dados na forma de um grafo. Existem diversos modelos de grafos que podem ser utilizados para este propósito, um exemplo de modelo que bastante utilizado é o dos grafos de propriedades, ou seja, grafos que possuem vértices, arestas e que tanto os vértices quanto as arestas podem possuir propriedades.
Assim, ao modelar os dados de um sistema em um banco de dados de grafos, os nós representam as entidades do sistema, enquanto as arestas representam as relações entre essas entidades, e as propriedades contém as demais informações sobre essas entidades e suas relações.

%inserir figura e um exemplo

Esses bancos de dados em geral possuem uma melhor escalabilidade do que os bancos de dados relacionais, já que não utilizam operações de junção nas consultas.
No entanto, sua maior vantagem é que, devido à sua estrutura, são capazes de realizar consultas próprias de grafos de maneira muito mais eficiente, como por exemplo encontrar o menor caminho entre dois nós, ou ainda descobrir se um dado nó é acessível à partir de outro, percorrendo o grafo apenas através de arestas que possuam uma determinada propriedade.

%comentar sobre os bancos existentes, livres ou não, etc?

\section{Métodos Ágeis}

\section{Desenvolvimento Orientado a Testes}\label{fun:tdd}

\subsection{Testes de Software}

\subsection{Complexidade Ciclomática}

\section{Arquitetura Cliente-Servidor}

\section{Sistemas Georreferenciados}

\section{Licenciamento de Software}

Licenças de software são instrumentos legais que regulam como um determinado software pode ser distribuído e utilizado.
Softwares, usualmente, não são vendidos, apenas tem seu uso licenciado, então a menos que o software seja colocado em domínio público, os direitos do usuário que licenciou o software e os do dono dos direitos autorais precisam ser bem definidos. Portanto, para que seus direitos autorais sejam respeitados, todo desenvolvedor de software deve se preocupar com o licenciamento de seu trabalho.

Além disso, todas as ferramentas de software citadas neste trabalho são distribuídas sob algum tipo de licença, e portanto é necessário conhecer e entender licenças de software para que se possa fazer um uso legítimo dessas ferramentas.

Existem dois tipos principais de licenças de software, as proprietárias e as livres e de código aberto. As licenças proprietárias se concentram em garantir que a distribuição do software seja potencialmente rentável, assim como proteger os direitos dos distribuidores. Já as licenças livres e de código aberto se concentram em proteger os direitos dos usuários.

\subsection{Software Proprietário} 

Software proprietário é aquele cuja licença restringe ou proíbe diversos usos do produto, como a cópia, redistribuição, engenharia reversa ou modificação. Além disso, o usuário que licencia o software normalmente não tem acesso ao código-fonte e a licença pode ter uma duração limitada.

Quando o usuário adquire uma cópia do software, esta costuma vir acompanhada da licença proprietária, na forma dos termos de uso do software. O software só pode ser utilizado se o usuário concordar com os termos de uso, e pode utilizá-lo apenas da forma prevista nos termos de uso.

\subsection{Software Livre e/ de Código Aberto}

Softwares livres e de código aberto são aqueles distribuídos com uma licença que permite que o usuário possa, além de utilizar, também analisar e melhorar o software, através do código fonte. Licenças de software livre e de código aberto tem duas definições diferentes, baseadas em conceitos parecidos, o de software livre e o de software de código aberto. Neste trabalho utilizaremos o o conceito de software livre.

Software livre, como definido pela \sigla{FSF}{Free Software Foundation}, é aquele cuja licença dá ao usuário 4 liberdades básicas\cite{GNUFreeSoftware}:

\begin{itemize}

\item A liberdade de executar o programa, para qualquer propósito (liberdade número 0).
\item A liberdade de estudar como o programa funciona, e adaptá-lo para as suas necessidades (liberdade número 1).
\item A liberdade de redistribuir cópias de modo que você possa ajudar o seu próximo (liberdade número 2).
\item A liberdade de distribuir cópias de sua versão modificada do programa para os outros (liberdade número 3).

\end{itemize}

Para satisfazer as liberdades 1 e 3, o usuário precisa ter acesso ao código fonte do software.

O objetivo dessa categoria de licenças é garantir ao desenvolvedor do software que qualquer um que receba o seu trabalho terá todas as liberdades listadas. Além disso, muitas licenças de software livre também são \emph{copyleft}, ou seja, exigem que trabalhos derivados mantenham todas as liberdades originais. Assim, algum terceiro fica impedido de, com base no trabalho original, realizar alguma modificação e em seguida distribuir a versão modificada sob uma licença proprietária, por exemplo.

%TODO: Listar/descrever licenças de software livre?

\section{Soluções Existentes}

\subsection{Soluções Proprietárias}

\subsection{Soluções Livres}

\section{Considerações}


