%----------- Capítulo 4: Metodologia --------------

\chapter{Metodologia}

% TODO: devemos falar da IDE usada? e do Jetty?

\section{Estratégia de Desenvolvimento do Software}

Tendo em vista que a equipe foi composta por apenas três integrantes, foi adotada uma estratégia de desenvolvimento de software inspirada nos Métodos Ágeis, prezando mais pela simplicidade e por um software funcional do que uma documentação extensa.

Para garantir a funcionalidade do software, foram periodicamente estabelecidas \emph{milestones} (ou metas) através de reuniões da equipe, cada uma destas correspondendo à uma funcionalidade específica do software.
Além disso, estas reuniões eram utilizadas para analisar a qualidade do código que havia sido escrito até então, em um processo chamado de \emph{code review} (ou revisão de código), bem como o cumprimento das \emph{milestones} previamente estabelecidas.
Ao estabelecer novas metas, cada integrante da equipe se responsabilizava por um sub-conjunto destas, de forma que cada um desenvolva uma parte do software que seja, de preferência, do seu interesse.

Para assegurar a qualidade do código escrito, foram utilizadas, além do processo de \emph{code review}, ferramentas de desenvolvimento orientado à testes e de cobertura dos testes, as quais são discutidas em maiores detalhes na seção a seguir.

\subsection{Desenvolvimento Orientado à Testes}

Para reforçar a idéia de qualidade do código do software, minimizar os \emph{bugs} e manter a simplicidade do código, optou-se por utilizar uma adaptação do processo de desenvolvimento de software orientado à testes.

% TODO: referenciar a seção da fundamentação teórica com um label
Como mencionado na fundamentação teórica, o processo de desenvolvimento de software orientado à testes baseia-se na repetição de um pequeno ciclo: inicialmente os desenvolvedores escrevem casos de teste nos quais o software falha, em seguida o código para que o software passe nos novos testes é escrito e, por fim, o código é refatorado para adequar-se aos padrões de código do projeto.

Neste projeto, a equipe optou por adaptar este método, de forma que não seja obrigatório escrever os casos de teste antes das funcionalidades do software, acelerando o processo de desenvolvimento.
No entanto, todo código escrito deve ser testado adequadamente antes de ser enviado ao sistema de controle de versão, com o objetivo de evitar que \emph{bugs} sejam introduzidos no código versionado.

O processo orientado à testes também permitiu que a equipe desenvolvesse o software em passos menores sem sacrificar a eficiência do grupo.
Os programadores apenas tinham se preocupar em cumprir as metas propostas na última reunião e garantir que o código passe nos testes, focando muito mais no desenvolvimento do código em si e nas funcionalidades desejadas do projeto.

Os casos de testes foram desenvolvidos por todos os membros da equipe, ou seja, não necessariamente por aqueles que desenvolveram uma determinada funcionalidade que seria testada pelos casos de teste em questão.
Este tipo de abordagem aumenta a confiabilidade dos testes, fazendo com que tanto o software quanto os testes tenham sido analisados por mais de uma pessoa, diminuindo a probabilidade de que erros surjam neste processo.


\subsubsection{Cobertura dos Testes}

Como no desenvolvimento orientado à testes os testes são usualmente escritos antes do código do software, não existe código escrito que não seja necessário para passar um teste.
Desta forma, os testes resultantes do processo de TDD acabam sempre, invariavelmente, aproximando-se de 100\% de cobertura de código durante os testes.
Ou seja, durante a execução dos testes, praticamente todo o código é percorrido pelos mesmos.

Devido à adaptação feita pela equipe à metodologia de desenvolvimento orientado à testes, a não-obrigatoriedade de escrever os testes antes do código, tornou-se necessário a utilização de uma métrica para analisar se os casos de teste escritos cobrem a maior parte do código escrito possível, com o objetivo de assegurar a qualidade dos testes.

Para analisar a cobertura dos testes escritos, a ferramenta \emph{Cobertura} foi utilizada, a qual é discutida em maiores detalhes na seção \ref{met:Cobertura}.


\subsection{Revisão de Código}

% TODO: citar wikipedia aos montes nessa seção

O processo de revisão de código é a examinação sistemática do código-fonte de um programa.
Este tem o objetivo de encontrar e corrigir erros cometidos durante o desenvolvimento, melhorando tanto a qualidade do software quanto as habilidades e conhecimentos dos programadores da equipe.

Os processos que podem ser utilizados para revisão de código encaixam-se em três categorias principais: programação em pares, revisão de código formal e a revisão de código leve.

% TODO: "tempo total de programador" nao parece ser um termo muito bom
A programação em pares, amplamente utilizada pelos métodos ágeis, corresponde à um conjunto de técnicas na quais dois programadores trabalham em uma única estação de trabalho: um dos programadores, o \emph{driver}, escreve o código, enquanto o outro, denominado \emph{observer}, revisa cada linha de código digitada.
Apesar de uma tarefa de programação ser concluída mais rapidamente quando dois programadores colaboram, estas técnicas possuem a desvantagem de aumentar o tempo total para o desenvolvimento do software. % TODO: justificar isso
Sendo assim, em uma equipe com menos integrantes como é o caso desta, o processo como um todo tornaria-se mais ineficiente, a equipe demoraria mais tempo para concluir o projeto e, por este motivo, a técnica de programação em pares não foi utilizada neste.

As técnicas de revisão de código formais são o método mais tradicional de revisão de código.
Estas envolvem um detalhado processo com múltiplos participantes e múltiplas fases, sendo o mais comum uma série de reuniões nas quais os desenvolvedores revisam o código linha por linha, geralmente utilizando cópias impressas.
Apesar de exigirem um grande investimento de tempo, tanto para a organização e planejamento quanto para a execução, as técnicas de revisão de código formais são consideradas efetivas para encontrar problemas no código sob revisão.

Por fim, as técnicas de revisão de código leves possuem um \emph{overhead} menor que as inspeções formais e são geralmente conduzidas como parte normal do processo de desenvolvimento de software.
Entre os principais exemplos deste tipo de técnica está a revisão de código auxiliada por ferramenta, na qual autores e revisores de código utilizam ferramentas especializadas para \emph{peer code review}, ou seja, revisão de código por pares.

Tendo em vista o tamanho da equipe, optou-se por utilizar a técnica de revisão de código auxiliada por ferramenta, sendo que após cada \emph{commit} no sistema de controle de versões todos os integrantes da equipe analisavam o que foi desenvolvido, fazendo comentários a respeito, sugerindo modificações e discutindo as decisões tomadas.
Além da revisão de código após os \emph{commits}, as reuniões da equipe também foram utilizadas para revisar o código escrito antes de um grande refatoramento no código do projeto ou de dar início ao desenvolvimento de uma nova importante funcionalidade do software.

Para o processo de revisão de código, foi utilizado o serviço de hospedagem de projetos de desenvolvimento de software GitHub, o qual conta com uma ferramenta de \emph{code review} que será apresentada em maiores detalhes na seção \ref{met:git}.


\section{Ferramentas Utilizadas}

\subsection{Linguagens de Programação}

Para os componentes do sistema que residem no servidor, ou seja, Core, GTFS Importer e Web Service, a equipe inicialmente optou por escolher uma linguagem de programação que todos os integrantes tivessem conhecimento e alguma experiência, para que não fosse necessário aprender uma nova linguagem, podendo então focar o tempo no aprendizado de outras ferramentas como SGBDs NoSQL e no desenvolvimento do código em si.
Desta forma, as linguagens que foram consideradas para este projeto foram: C++, Java e Python.

% TODO: colocar uma tabela comparativa das linguagens C++, Java e Python

A linguagem C++ foi criada por Bjarne Stroustrup em 1979 no Bell Labs.
Originalmente, esta foi criada como uma extensão que insere diversas funcionalidades à linguagem C, tais como classes, sobrecarga de operadores, herança múltipla, \emph{templates} e tratamento de exceções.
Atualmente, C++ é uma das linguagems mais populares, com domínios de aplicações que incluem sistemas operacionais, aplicativos para \emph{desktops}, \emph{drivers} para dispositivos de hardware, sistemas embarcados e aplicações de servidores e clientes que exigem alta performance.
% TODO: talvez falar um pouco sobre a performance de C++ ser bem melhor que a de Python e Java

O principal motivo pelo qual a equipe optou por descartar a linguagem C++ foi a ausência de um sistema de \emph{garbage collection}, o que torna mais difícil a manutenção da base de código e aumenta a probabilidade de erros de programação relacionados com \emph{memory leaks} (ou ``vazamentos de memória''), que são um problema comum quando uma porção de memória alocada para uma determinada operação não é liberada quando não é mais necessária.
Além disso, ao contrário da linguagem C++, as linguagens Java e Python possuem ferramentas para desenvolvimento de aplicações web amplamente utilizadas e documentadas, como os \emph{frameworks} Django e Pylons da linguagem Python e a plataforma J2EE da linguagem Java.

A linguagem Python, por sua vez, foi criada por Guido van Rossum no fim da década de 80, sendo que a primeira implementação do interpretador CPython, a implementação referência da linguagem, começou a ser desenvolvida em Dezembro de 1989.
A filosofia desta linguagem enfatiza a legibilidade do código, reforçando uma sintaxe clara e de fácil entendimento até mesmo para aqueles que não conhecem a linguagem.
Python também possui uma biblioteca padrão bastante completa, que é considerada como um de seus pontos mais fortes, com módulos que vão desde aritmética com precisão arbitrária até expressões regulares, interfaces com bancos de dados relacionais e execução de testes unitários.

A linguagem Java foi criada por James Gosling enquanto trabalhava na Sun Microsystems.
Lançada em 1995 como um dos principais componentes da plataforma Java, a linguagem teve sua sintaxe inspirada na de C++, compartilhando diversas características com a mesma.
Ao contrário de C++, os aplicativos desenvolvidos em Java são compilados na forma de \emph{bytecode} e podem, portanto, ser executados em qualquer computador que possua uma máquina virtual Java instalada, independente de arquitetura ou sistema operacional.
Assim como Python, a linguagem Java possui uma extensa biblioteca padrão, também contando com módulos para acesso à bancos de dados relacionais, \emph{toolkits} para a confecção de interfaces gráficas, \emph{sockets}, \emph{threads}, entre outros.
Outra vantagem da linguagem Java são os Servlets: classes Java que estendem as funcionalidades de servidores que hospedam aplicações acessadas por meio de requisições e respostas.
Apesar de poderem ser utilizadas para responder qualquer tipo de requisição, os Servlets são geralmente utilizados para estender aplicações hospedadas por servidores Web e, sendo assim, de grande utilidade para este projeto.

A decisão pela linguagem de programação a ser utilizada nas componentes da aplicação presentes no servidor foi influenciada em grande parte pela decisão do Sistema Gerenciador de Banco de Dados do projeto.
Como é apresentado na Tabela \ref{tab:bancos}, os principais SGBDs NoSQL existentes são escritos em linguagem Java e, sendo assim, todos estes possuem \emph{bindings} para esta.
Além disso, os integrantes da equipe possuíam conhecimentos prévios de linguagem Java, assim como construção de Servlets.
Somando-se estes pontos à extensa documentação fornecida pela linguagem, optou-se por utilizar a linguagem Java nas aplicações do servidor.

Com relação ao cliente Web, foram consideradas duas opções: a primeira seria desenvolver este utilizando a linguagem Java, sob a forma de um Applet Java, enquanto que a segunda seria utilizar apenas utilizar páginas em HTML simples com o auxílio de JavaScript para melhorar a experiência do usuário.

A principal desvantagem de se utilizar Applets Java para a interface do cliente Web é a necessidade do usuário ter uma implementação da máquina virtual Java instalada em seu computador para utilizar o sistema.
Além disso, devido ao \emph{overhead} imposto pela máquina virtual, os Applets Java tendem a demorar mais tempo para carregar no navegador Web dos usuários, prejudicando a experiência do usuário.
Sendo assim, esta opção foi descartada pela equipe e optou-se por desenvolver o cliente Web utilizando apenas páginas HTML e JavaScript.

A linguagem JavaScript foi desenvolvida por Brendan Eich durante a década de 90, enquanto trabalhava na Netscape.
Apesar do nome, a linguagem não possui grande relação com Java, excetuando-se o fato de que aquela foi influenciada por esta em alguns aspectos como, por exemplo, as convenções de nome e a de sintaxe.
Atualmente, JavaScript é amplamente utilizada para o desenvolvimento de interfaces de usuário para a \emph{Web} e \emph{websites} dinâmicos.

Com o crescimento da demanda por aplicações escritas em JavaScript, uma série de bibliotecas para facilitar o desenvolvimento deste tipo de aplicação surgiu.
Entre estas destaca-se a biblioteca jQuery que, por ser a mais popular, contém uma extensa documentação e diversas funcionalidades, que vão desde efeitos e animações, até manipulação de folhas de estilo e métodos para comunicação assíncrona com servidores de aplicações Web.
Por estes motivos, optou-se por utilizar também esta biblioteca para o desenvolvimento do cliente Web.

\subsection{Sistema Gerenciador de Banco de Dados}

% TODO: falar algo do Neo4j spatial

Como citado anteriormente, os SGBDs NoSQL permitem que os dados sejam armazenados de uma forma menos rígida que nos bancos de dados relacionais.
Por este motivo, os bancos de dados NoSQL apresentam uma grande vantagem com relação aos demais para armazenar e extrair informações rapidamente de bases de dados que possam ser modeladas como grafos. % TODO: citar algo

Dada a natureza do problema sendo estudado, naturalmente a equipe optou por utilizar um dos bancos de dados NoSQL que enfatizam em grafos.
Entre as principais opções disponíveis que foram consideradas para este projeto estão o HyperGraphDB, desenvolvido pela Kobrix Software Inc., o InfoGrid, desenvolvido pela NetMesh Inc., o Neo4j, desenvolvido pela Neo Technology Inc. e o OrientDB, desenvolvido pela Orient Technologies.
Estes bancos de dados são apresentados na Tabela \ref{tab:bancos}.

Outros bancos de dados disponíveis na Internet também foram inicialmente analisados, como o FlockDB, desenvolvido pela Twitter Inc..
No entanto, estes foram desconsiderados posteriormente por não possuírem documentação suficiente no momento em que foi dado início ao desenvolvimento do projeto ou focarem demais em um problema específico, como é o caso do próprio FlockDB, que é utilizado para armazenar as relações sociais entre os usuários do serviço do Twitter: ``quem segue quem'' e ``quem é seguido por quem''.

\begin{table}[!htb]
	\centering
	\caption{Tabela comparativa entre as opções de SGBDs disponíveis}
	\label{tab:bancos}
	\begin{tabular}{lcccc}
		\hline
		& \textbf{HyperGraphDB} & \textbf{InfoGrid} & \textbf{Neo4j} & \textbf{OrientDB} \\
		\hline
		\textbf{Licença} & LGPL & AGPLv3 & AGPLv3 & Apache \\
		\textbf{Iniciado em} & 2005 & ? & 2003 & ? \\
		\textbf{Versão estável} & 1.1 & 2.9.5 & 1.4.2 & 0.9.25 \\
		\textbf{Data versão estável} & Dezembro 2010 & Agosto 2011 & Setembro 2011 & Março 2011 \\
		\textbf{Bindings Java} & Sim & Sim & Sim & Sim \\
		\textbf{Bindings Python} & Não & Não & Sim & Parcial \\
		\textbf{Bindings C/C++} & Não & Não & Não & Não \\
		\textbf{Stand-alone} & Sim & ? & Sim & Sim \\
		\textbf{Embarcado} & Não & ? & Sim & Sim \\
		\textbf{Suporte Blueprints} & Não & Não & Sim & Sim \\
		\hline
	\end{tabular}
	\fonte{Autoria pr\'opria.}
\end{table}

Entre os SGBDs apresentados na Tabela \ref{tab:bancos}, todos são desenvolvidos em linguagem Java e, portanto rodam em qualquer plataforma que possua uma implementação da \sigla{JVM}{Java Virtual Machine}.
Além disso, todos eles são disponibilizados através de uma licença de software livre.
No entanto, os bancos de dados InfoGrid e Neo4j são também distribuídos através de licenças comerciais, sendo a versão livre destes uma versão com menos funcionalidades e sob uma licença, a \sigla{AGPLv3}{Affero General Public License Version 3}, que impede o desenvolvedor de não disponibilizar o código-fonte de um software que utilize o banco de dados. % TODO: verificar se isso que eu falei da AGPLv3 ta certo

Todos os bancos de dados que foram levados em consideração para este projeto apresentavam um conjunto de funcionalidades semelhantes.
Como todos são desenvolvidos em linguagem Java, todos possuem bindings para esta linguagem, reforçando então ainda mais a escolha por esta linguagem.

Desta forma, os fatores determinantes para a decisão sobre qual banco de dados utilizar, acabaram sendo o suporte dado pela comunidade do mesmo, a documentação e quão frequentes são as atualizações do software, tendo em vista que aqueles que são mais bem atualizados possuem uma comunidade mais ativa.
O HyperGraphDB foi desconsiderado por não ser atualizado com frequência, dado que sua última versão estável era de Dezembro de 2010, muito anterior com relação às dos demais bancos.

Já a documentação e suporte da comunidade dos bancos OrientDB e Neo4j mostrou-se muito superior à do InfoGrid, tendo em vista que estes possuem um volume maior de emails nas suas listas de discussões e, portanto, uma comunidade mais ativa.

A decisão final de qual banco de dados utilizar ficou, portanto, entre o Neo4j e o OrientDB.
Optou-se pelo Neo4j principalmente devido à facilidade de aprendizado da sua \sigla{API}{Application Programming Interface}, que possui uma notação mais concisa e com uma curva de aprendizagem adequada, na opinião da equipe.
Outro ponto positivo com relação ao Neo4j é que ele é mantido por uma grande comunidade de desenvolvedores desde 2003 e é, portanto, um dos projetos livres pioneiros nesta área.

\subsection{Gerenciamento e Automação de Projetos}

Tendo em vista a arquitetura em módulos proposta para este projeto, mostrou-se interessante o uso de uma ferramenta capaz de facilitar o processo de \emph{build}, ou seja, compilação e integração dos diferentes módulos da aplicação.

Para tanto, optou-se por utilizar o Apache Maven, o qual é uma ferramenta desenvolvida em linguagem Java para facilitar o gerenciamento de projetos nesta e em outras linguagens de programação.

Entre as principais vantagens de se utilizar o Maven para o gerenciamento de um projeto, estão:
\begin{itemize}
	\item Facilita o processo de \emph{build}, compilação e integração dos módulos da aplicação.
	\item Simplifica a resolução de dependências, permitindo que o desenvolvedor apenas especifique de quais bibliotecas ou módulos o projeto depende, deixando o trabalho de fazer \emph{download} das mesmas e de inserir no \emph{classpath} à cargo do Maven.
	\item Permite a execução de testes unitários e de integração no projeto antes de construir um pacote do mesmo.
	\item Facilita o gerenciamento de projetos de grande porte.
	% TODO: citar mais vantagens
\end{itemize}

Os projetos gerenciados pelo Apache Maven são descritos através de uma estrutura denominada \sigla{POM}{Project Object Model} (\emph{Project Object Model}).
Esta estrutura, especificada sob a forma de um documento \sigla{XML}{Extensible Markup Language} denominado \texttt{pom.xml}, fornece à ferramenta todas as informações relativas ao projeto como, por exemplo: o nome do projeto, os desenvolvedores, o tipo de empacotamento utilizado para a aplicação e suas dependências com outros projetos.
Através do POM também é possível configurar todas as fases do processo de compilação, permitindo especificar até mesmo a versão da linguagem Java que será utilizada para compilação ou evitar que seja gerado um pacote da aplicação quando algum dos testes falhar.

Projetos maiores, como é o caso deste, podem ser divididos em diversos módulos, ou sub-projetos, cada um com o seu próprio POM.
Um projeto principal (ou raiz) é criado, através do qual é possível compilar todos os demais módulos com um único comando.
Os POMs dos sub-projetos podem ainda herdar configurações do projeto principal, tornando possível manter uma padronização entre os projetos e sub-projetos da aplicação com facilidade.

Dadas estas características, esta ferramenta mostrou-se ideal para o desenvolvimento deste projeto, tendo em vista a grande variedade de dependências que compõem este, tais como bibliotecas para acesso ao SGBD, serialização e deserialização de dados, assim como os diversos módulos pelos quais a aplicação é composta.

% TODO: talvez seja interessante ampliar essa seção, dando um exemplo de pom.xml, mais detalhes sobre a ferramenta e tal

\subsection{Sistema de Controle de Versão}\label{met:git}

O gerenciamento e monitoramento de modificações em documentos, programas e outras informações armazenadas como arquivos de computador é conhecido como controle de versão ou, ainda, controle de revisão.

Durante o processo de desenvolvimento de software, é comum que múltiplas versões de um mesmo software sejam utilizadas em diferentes locais e que os desenvolvedores trabalhem simultaneamente em partes distintas do software.
Novas funcionalidades ou até mesmo \emph{bugs} muitas vezes estão presentes apenas em determinadas versões.
Portanto, para organizar o desenvolvimento, bem como facilitar a localização e correção de \emph{bugs}, é de vital importância determinar em quais versões do \emph{software} os problemas ocorrem.
Inclusive, muitas vezes é necessário desenvolver duas ou mais versões do mesmo software em paralelo como, por exemplo, uma na qual os problemas sejam corrigidos sem a criação de novas funcionalidades, enquanto que em outra versão novas funcionalidades são inseridas na aplicação, sendo estas comumente denominadas \emph{branch} e \emph{trunk}, respectivamente.

Enquanto os desenvolvedores poderiam simplesmente manter múltiplas cópias das diferentes versões do projeto no sistema de arquivos, este processo é extremamente ineficiente pois muitas versões praticamente idênticas do \emph{software} teriam que ser mantidas, tornando a tarefa de determinar as modificações entre uma versão e outra difícil, tediosa e sujeita à erros.
Por este motivo, diversos sistemas para automatizar o processo de controle de versão surgiram.

Tradicionalmente, os sistemas de controle de versão mais cmouns utilizam um modelo centralizado, no qual todas as funções são desempenhadas em um servidor compartilhado por todos os desenvolvedores.
Sistemas como estes vêm sendo utilizados desde a década de 70 e foram considerados padrão na indústria por muitos anos.
Diversos sistemas de controle de versão livres enquadram-se neste modelo como, por exemplo, o \sigla{CVS}{Concurrent Versions System}, lançado em 1990, e o Subversion, lançado em 2004.

\subsection{Testes e Cobertura de Código}\label{met:Cobertura}

\section{Considerações}

