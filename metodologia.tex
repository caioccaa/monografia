%----------- Capítulo 4: Metodologia --------------

\chapter{Metodologia}

\section{Estratégia de Desenvolvimento do Software}

\subsection{Desenvolvimento Orientado à Testes}

\subsubsection{Cobertura dos Testes}

\subsection{Revisão de Código}

\section{Ferramentas Utilizadas}

\subsection{Java}

\subsection{Banco de Dados}

Como citado anteriormente, os bancos de dados NoSQL permitem que os dados sejam armazenados de uma forma menos rígida que nos bancos de dados relacionais.
Por este motivo, os bancos de dados NoSQL apresentam uma grande vantagem com relação aos demais para armazenar e extrair informações rapidamente de bases de dados que possam ser modeladas como grafos. % TODO: citar algo

Dada a natureza do problema sendo estudado, naturalmente a equipe optou por utilizar um dos bancos de dados NoSQL que enfatizam em grafos.
Entre as principais opções disponíveis que foram consideradas para este projeto estão o HyperGraphDB, desenvolvido pela Kobrix Software Inc., o InfoGrid, desenvolvido pela NetMesh Inc., o Neo4j, desenvolvido pela Neo Technology Inc. e o OrientDB, desenvolvido pela Orient Technologies.
Estes bancos de dados são apresentados na Tabela \ref{tab:bancos}.

Outros bancos de dados disponíveis na Internet também foram inicialmente analisados, como o FlockDB, desenvolvido pela Twitter Inc..
No entanto, estes foram desconsiderados posteriormente por não possuírem documentação suficiente no momento em que foi dado início ao desenvolvimento do projeto ou focarem demais em um problema específico, como é o caso do próprio FlockDB, que é utilizado para armazenar as relações sociais entre os usuários do serviço do Twitter: ``quem segue quem'' e ``quem é seguido por quem''.

\begin{table}[!htb]
	\centering
	\caption{Tabela comparativa entre as opções de Bancos de Dados analisadas}
	\label{tab:bancos}
	\begin{tabular}{lcccc}
		\hline
		& \textbf{HyperGraphDB} & \textbf{InfoGrid} & \textbf{Neo4j} & \textbf{OrientDB} \\
		\hline
		\textbf{Licença} & LGPL & AGPLv3 & AGPLv3 & Apache \\
		\textbf{Iniciado em} & 2005 & ? & 2003 & ? \\
		\textbf{Versão estável} & 1.1 & 2.9.5 & 1.4.2 & 0.9.25 \\
		\textbf{Data versão estável} & Dezembro 2010 & Agosto 2011 & Setembro 2011 & Março 2011 \\
		\textbf{Bindings Java} & Sim & Sim & Sim & Sim \\
		\textbf{Bindings Python} & Não & Não & Sim & Parcial \\
		\textbf{Bindings C/C++} & Não & Não & Não & Não \\
		\textbf{Stand-alone} & Sim & ? & Sim & Sim \\
		\textbf{Embarcado} & Não & ? & Sim & Sim \\
		\textbf{Suporte Blueprints} & Não & Não & Sim & Sim \\
		\hline
	\end{tabular}
	\fonte{Autoria pr\'opria.}
\end{table}

Entre os bancos de dados apresentados na Tabela \ref{tab:bancos}, todos são desenvolvidos em linguagem Java e, portanto rodam em qualquer plataforma que possua uma implementação da \sigla{JVM}{Java Virtual Machine}.
Além disso, todos eles são disponibilizados através de uma licença de software livre.
No entanto, os bancos de dados InfoGrid e Neo4j são também distribuídos através de licenças comerciais, sendo a versão livre destes uma versão com menos funcionalidades e sob uma licença, a \sigla{AGPLv3}{Affero General Public License Version 3}, que impede o desenvolvedor de não disponibilizar o código-fonte de um software que utilize o banco de dados. % TODO: verificar se isso que eu falei da AGPLv3 ta certo

Todos os bancos de dados que foram levados em consideração para este projeto apresentavam um conjunto de funcionalidades semelhante.
Como todos são desenvolvidos em linguagem Java, todos possuem bindings para esta linguagem, reforçando então ainda mais a escolha por esta linguagem.

Desta forma, os fatores determinantes para a decisão sobre qual banco de dados utilizar, acabaram sendo o suporte dado pela comunidade do mesmo, a documentação e quão frequentes são as atualizações do software, tendo em vista que aqueles que são mais bem atualizados possuem uma comunidade mais ativa.
O HyperGraphDB foi desconsiderado por não ser atualizado com frequência, dado que sua última versão estável era de Dezembro de 2010, muito anterior com relação às dos demais bancos.

Já a documentação e suporte da comunidade dos bancos OrientDB e Neo4j mostrou-se muito superior à do InfoGrid, tendo em vista que estes possuem um volume maior de emails nas suas listas de discussões e, portanto, uma comunidade mais ativa.

A decisão final de qual banco de dados utilizar ficou, portanto, entre o Neo4j e o OrientDB.
Optou-se pelo Neo4j principalmente devido à facilidade de aprendizado da sua \sigla{API}{Application Programming Interface}, que foi mais fácil de ser entendida, na opinião da equipe.
Outro ponto positivo com relação ao Neo4j é que ele é mantido por uma grande comunidade de desenvolvedores desde 2003 e é, portanto, um dos projetos livres pioneiros nesta área.

\subsection{Maven}

\subsection{git}

\subsection{JUnit}

\subsection{Cobertura}

\section{Considerações}

