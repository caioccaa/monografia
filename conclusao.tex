\chapter{Conclusão}

Como produto do projeto em questão obteve-se uma ferramenta disponibilizada na forma de software livre através de um serviço web, com a principal funcionalidade de auxiliar o planejamento de trajetos urbanos com o uso de transporte público.
O código-fonte do projeto inteiro está disponível livremente para o público, basta seguir os passos citados no apêndice \ref{guia}.
Desta forma, o projeto está aberto a futuras contribuições, adaptações e integrações a diversos outros sistemas, não sendo limitado somente a sua funcionalidade originalmente idelizada pela equipe.


% TODO: Discutir a metodologia utilizada (se foi bom e tal)


% TODO: Cada uma das considerações dos capítulos deverão entrar na conclusão
%especificação
O passo inicial do projeto foi uma profunda revisão bibliográfica, como o intuito de analisar todas as tecnologias existentes e definir qual seria a mais indicada para o desenvolvimento do software, não somente em questão de programação mas também em decisões de metodologias e técnicas de gerência de projeto utilizadas.

