\apendice
\chapter{Exemplo de Uso do \emph{Service}}\label{ape:exemplodeuso}

Este apêndice procura fornecer um claro exemplo de como executar consultas através do componente \emph{Service} do projeto dentro de uma aplicação.
O código apresentado a seguir, em linguagem Python, executa uma consulta no serviço, decodifica o JSON do objeto resposta e, por fim, imprime o resultado na saída padrão.

Vale lembrar que, para executar este exemplo, é necessário ter instalado o interpretador da linguagem Python na versão 2.6 ou superior.

% TODO: falar que o endereço de acesso do service depende do servidor
% falar também que a URL do RouteServlet depende do especificado no arquivo web.xml do onibuscerto-service/src/main/webapp/WEB-INF

% TODO: será que esse exemplo de cliente em Python fica nessa seção mesmo?
% talvez seja melhor colocar me um anexo ou coisa parecida
% outra coisa, não sei como referenciar ele no texto, então ficou assim msm
\lstinputlisting[language=Python]{code/cliente.py}
